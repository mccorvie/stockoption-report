
% From http://blog.poormansmath.net/latex-class-for-lecture-notes/

% The options (see the first line) are these:

% english or italian: the document’s language;
% course, seminar or talk: a course is a medium-length document structured in sections, subsections and paragraphs; a seminar is a short document without structure or with subsections; a talk is the sheets you take with you when you give a seminar.

% The second option defines the style of the document; if you want to tweak some parameters, there are also these fine-tuning options:

% headertitle, headersection, headersubsection or headerno: what to display on the top of the pages;
% theoremnosection, theoremsection or theoremsubsection: numbering of theorems, definitions, etc.;
% cleardoublepage or nocleardoublepage: whether we want empty double pages after a section ending;
% oneside or twoside: margins and headers optimized for one-sided or two-sided printing;
% onecolumn or twocolumn: how many columns per page.

\documentclass[english,seminar]{pnotes}


% Uncomment to use Ryan's frequently used macros
\RequirePackage{Ryanalysis}
\RequirePackage{Ryanprob}


% Obviously you have to replace the values, or, when appropriate, delete some line. Note that “author” and “email” are name and e-mail of the notes taker. 

\title{Black Scholes Notes}
\subtitle{Expectations}
% \subject{SUBJECT}
\author{Ryan McCorvie}
% \email{EMAIL}
\date{01}{01}{2017}
% \dateend{DD}{MM}{YYYY}
% \place{PLACE}



% The class defines also some commands:

% for course, the command \lecture{duration}{dd}{mm}{yyyy} which writes on the margin information about the lesson’s number, duration and date;
% for talk, the commands \separator, which simply draws a horizontal line, and \tosay{message} which prints the message inside a box (useful to remember things you don’t have to write on the blackboard);
% \mymarginpar{message}, which behaves like a regular \marginpar but with a custom style.

\begin{document}
\subsection{Normal options}
Suppose $X\sim \dnorm(\mu,\sigma^2)$.  Then
\begin{equation}
\begin{split}
	\E (X-K)^{+} &= \frac 1 {\sigma \sqrt{2 \pi}} \int_{\R} (x-K)^{+} e^{-\frac 1 2 \left( \frac{x-\mu} \sigma \right)^2} \, dx \\
	&= \frac 1 { \sqrt{2 \pi}}\int_{\R} (\sigma u + \mu - K)^{+} e^{-\frac 1 2 u^2} \, du \\
	&= \frac 1 { \sqrt{2 \pi}}\int_{(K-\mu)/\sigma}^\infty (\sigma u + \mu - K) e^{-\frac 1 2 u^2} \, du \\
	&= (\mu-K) \Phi\left( \frac{ \mu-K} \sigma \right)  + \frac \sigma { \sqrt{2 \pi}} e^{- \frac 1 2 \left(\frac{\mu-K} \sigma \right)^2}
\end{split}
\end{equation}
If $\sigma = s\sqrt{t} $ and $\E X = \mu = F$ is the forward price, then this can be written
\begin{equation}
	\E(X-K)^+ = (F-K) \, \Phi( d  ) + \frac {s\sqrt{t}  }{ \sqrt{2 \pi}} e^{- \frac 1 2 d^2} \qquad \text{where} \qquad d = \frac{F-K}{s\sqrt{t} }
\end{equation}

\subsection{Lognormal options}

Now suppose $Y=e^{X}$ where $Y\sim \dnorm(\mu,\sigma)$.  Then
\begin{equation}
\begin{split}
	 \E (Y-K)^+ &=\frac 1 {\sigma \sqrt{2 \pi}} \int_{\R} (e^x - K)^+ \, e^{-\frac 1 2 \left(\frac{x-\mu}{\sigma}\right)^2}\, dx\\
	 &= \frac 1 {\sqrt{2 \pi}} \int_{\R} (e^{\sigma u + \mu} - K)^+ \, e^{-\frac 1 2 u^2}\, du\\
	 &= \frac 1 {\sqrt{2 \pi}} \int_{(\log K-\mu)/\sigma}^\infty (e^{\sigma u + \mu}-K) \,e^{-\frac 1 2 u^2}\, du\\
	 &= -K \, \Phi\left(\frac{\mu - \log K} \sigma \right) + e^{\mu + \frac 1 2 \sigma^2} \frac 1 {\sqrt{2 \pi}}  \int_{(\log K-\mu)/\sigma}^\infty e^{-\frac 1 2 (u-\sigma)^2}\, dx \\
	 &= -K \, \Phi\left(\frac{\mu - \log K} \sigma \right) + e^{\mu+\frac 1 2 \sigma^2} \Phi\left( \frac{\mu - \log K  + \sigma^2} \sigma \right)
 \end{split}
\end{equation}
When $K \to -\infty$, then $\E(Y-K)^+ \to \E Y = e^{\mu + \frac 1 2 \sigma^2}$.  If we write $F= \E Y$ and $\sigma = s \sqrt t$, this can be written more symetrically as
\begin{equation}
	\E(Y-K)^+ = F \, \Phi(d_+) -K \, \Phi( d_- ) \label{eq:black scholes}
\end{equation}
where
\begin{equation}
	d_- = \frac{\log(F/K) -\frac 1 2 s^2 t}{s\sqrt t} \qquad  d_+ = d_+ + s \sqrt t  = \frac{\log(F/K) +\frac 1 2 s^2 t}{s \sqrt t}	
\end{equation}
Let the forward call price be given by $c_F = \E(Y-K)^+$ and the forward put price be given by $p_F = \E(K-Y)^+$.  Then put call pairity is
\begin{equation}
	c_F - p_F = \E(Y-K) = F-K
\end{equation}
Thus
\begin{equation}
	p_F = \E(K-Y)^{+} = -F\,\Phi(-d_+) + K \, \Phi(-d_-)
\end{equation}

\subsection{Greeks}

Let's compute sensitivties.  Also, assume the forward price is given by $e^{(r-q)t}S$, where $r$ is the riskfree interest rate and $q$ is the convenience yield or dividend yield.  As before, let the forward call price be given by $c_F = \E(X-K)^+$  and the spot option price be given by $c = e^{-rt} c_F$, and let's compute the generic partial derivative with respect ot $u$-- we'll specialize later.  

\begin{equation}
	\frac {\partial c_F}{\partial u} = \frac{\partial F}{\partial u} \, \Phi(d_+) + F \Phi'(d_+) \frac{\partial d_+}{\partial u} - \frac{\partial K} {\partial u} \, \Phi(d_-) - K \Phi'(d_-) \frac{\partial{d_-}}{\partial u} \label{eq:greek1}
\end{equation}
Note that 
\begin{equation}
	d_+^2-d_-^2 = (d_++d_-)(d_+-d_-)= \frac{2 \log(F/K)}{s \sqrt t} \cdot s \sqrt t = 2\log(F/K)
\end{equation}
Therefore, we can define
\begin{equation}
	\Psi := F \Phi'(d_+) = \frac F {\sqrt{2\pi}} e^{-\frac 1 2 d_+^2} = \frac K {\sqrt{2\pi}}  e^{-\frac 1 2 d_-^2}= K \Phi'(d_-)
\end{equation}
This allows us to simplify \eqref{eq:greek1}
\begin{equation}
	\frac {\partial c}{\partial u} = \frac{\partial F}{\partial u} \, \Phi(d_+)  - \frac{\partial K} {\partial u} \, \Phi(d_-) + \Psi  \, \frac{\partial (d_+-d_-)}{\partial u} \label{eq:greek master}
\end{equation}
First let's consider sensitivities to variables on which $d_+-d_- = s \sqrt t$ does not depend, so the last term is 0.  
\begin{equation}
\begin{split}
	m = \frac{\partial c_F}{\partial K} &= -\Phi(d_-) = -\Pr( Y \geq K ) \\
	\Delta_F = \frac{\partial c_F}{\partial S} &= \frac{\partial F}{\partial S} \Phi(d_+)  = e^{(r-q)t} \Phi(d_+)\\
	\Delta = \frac{\partial c}{\partial S} &= e^{-rt} \frac{\partial c_F}{\partial S}  = e^{-q t} \Phi(d_+)\\
	\rho_F = \frac{\partial c_F}{\partial r} &= \frac{\partial F}{\partial r} \Phi(d_+) = t F \Phi(d_+) \\
	\rho = \frac{\partial c}{\partial r} &= -t c + e^{-rt} t F \Phi(d_+) = t K \Phi(d_-)
\end{split}
\end{equation}
The quantity $m$ is sometimes called moneyness.  Now let's compute greeks which do include the last term.  Define a new quantity
\begin{equation}
	\Upsilon = e^{-rt}\Psi = e^{-rt} K \Phi'(d_-) = e^{-qt} S \Phi'(d_+)	
\end{equation}
Then we can compute
\begin{equation}
\begin{split}
	\text{vega}_F = \frac{\partial c_F}{\partial s} &=  \sqrt t \Psi\\
	\text{vega} = \frac{\partial c}{\partial s} &=  e^{-r t}  \frac{\partial c_F}{\partial s} = \sqrt t \Upsilon \\
	\tau_F = -\frac{\partial c_F}{\partial t} &=  (q-r)F \Phi(d_+) - \frac {s \Psi} {2 \sqrt{t}} \\
	\tau = -\frac{\partial c}{\partial t} &= r c - e^{-rt}\frac{\partial c_F}{\partial t} = q F \Phi(d_+) - rK \Phi(d_-) - \frac {s \Upsilon } {2 \sqrt{t}}
\end{split}
\end{equation}
Finally we compute
\begin{equation}
\begin{split}
	\Gamma_F = \frac{\partial^2 c_F}{ \partial S^2} &= \frac{F}{S} \Phi'(d_+) \frac{\partial d_+}{\partial S}  = \frac {\Psi}{ S^2 s \sqrt{t}} \\
	\Gamma = \frac{\partial^2 c}{ \partial S^2} &= e^{-rt} \Gamma_F = \frac {\Upsilon}{ S^2 s \sqrt{ t}}
\end{split}
\end{equation}
\end{document}
